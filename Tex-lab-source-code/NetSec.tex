\documentclass[aspectratio=169]{beamer}

\usepackage{tikz}
\usepackage[T1]{fontenc}
\usepackage{newunicodechar}
\usepackage{listings}
\usepackage{comment}
\usepackage{amsmath}
\usepackage[]{graphicx}
\usepackage{subfig}
\usepackage{marvosym}
\usepackage{hyperref}
\usepackage{minted}
\usetikzlibrary{positioning}
\usetheme{ulisse}
\usepackage{listings}
\usepackage{xcolor}


\definecolor{red2}{HTML}{ED4F4F}

% redefinition of \texttt
\let\Oldtexttt\texttt
\renewcommand\texttt[1]{{\ttfamily\color{red2}#1}}
\makeatother
\lstset{ 
    % language=Cy, % choose the language of the code
    basicstyle=\fontfamily{pcr}\selectfont\footnotesize\color{white},
    keywordstyle=\color{black}\bfseries, % style for keywords
    numbers=none, % where to put the line-numbers
    numberstyle=\tiny, % the size of the fonts that are used for the line-numbers     
    backgroundcolor=\color{black},
    showspaces=false, % show spaces adding particular underscores
    showstringspaces=false, % underline spaces within strings
    showtabs=false, % show tabs within strings adding particular underscores
    frame=single, % adds a frame around the code
    tabsize=2, % sets default tabsize to 2 spaces
    rulesepcolor=\color{gray},
    rulecolor=\color{black},
    captionpos=b, % sets the caption-position to bottom
    breaklines=true, % sets automatic line breaking
    breakatwhitespace=false, 
}


\usepackage{hyperref} % link color
\hypersetup{
    colorlinks=true,
    linkcolor=black,
    urlcolor=red2,
}

\title{Network Security -- Laboratorio}
\author{Karina Chichifoi (@TryKatChup)}
\date{\today}

\begin{document}
 	\maketitle

	% Intro
	\begin{frame}
		\frametitle{Monitoraggio della rete}
		Per rilevare gli attacchi che arrivano alla nostra rete si guarda il traffico in ingresso e in uscita. Occorre quindi un sistema che:
		\begin{itemize}
		    \item esamini tutto il traffico senza rallentarlo
		    \item generi pochissimi falsi allarmi (quindi pochi falsi positivi)
		    \item non lasci sfuggire attacchi reali (falsi negativi nulli)
		\end{itemize}
	\end{frame}

	\begin{frame}
		\frametitle{IDS e IPS}
	    \begin{itemize}
	        \item IDS (Intrusion Detection System): rilevano attività inappropriate, errate o anomale in una rete e le segnalano.
	        \item IPS (Intrusion Prevention System): è una componente hardware o software il cui scopo è quello di prevenire tentativi di attacco per gli host che fanno parte della rete.
	    \end{itemize}
	    Ad esempio un IPS può effettuare il drop dei pacchetti, il reset di una sessione o bloccare e blacklistare un host che sta eseguendo un attacco alla rete.
	\end{frame} 
    
    \begin{frame}
        \frametitle{NIDS VS HIDS}
        Esistono due tipologie di IDS:
        \begin{itemize}
            \item NIDS (Network Intrusion Detection System): i sistemi esaminano il traffico nella rete e monitorano più host per identificare le intrusioni;
            \item HIDS (Host Intrusion Detection System): analizzano i dati e le attività di un singolo host, come ad esempio i registri di sistema, i log, modifiche al file system e così via.
        \end{itemize}
    \end{frame}
    
    
	\begin{frame}
		\frametitle{NIDS}
        La rilevazione di attacchi che giungono via rete viene svolta analizzando il traffico in entrata/uscita.
        Problema essenziale:
        \begin{itemize}
            \item esaminare tutto il traffico senza rallentarlo
            \item generando pochissimi falsi allarmi
            \item senza lasciar sfuggire attacchi reali
        \end{itemize}
        
        Due approcci:
        \begin{itemize}
            \item Signature based: rileva flussi con caratteristiche notoriamente
        malevole
            \item Anomaly based: rileva flussi che si discostano dalla “normalità”
        \end{itemize}
        
        Tra i NIDS più diffusi abbiamo
        \begin{itemize}
            \item Snort
            \item Suricata
            \item Bro
        \end{itemize}
	\end{frame}
	
	\begin{frame}
        \frametitle{Suricata}
        \begin{tikzpicture}[overlay,remember picture]
            \node[anchor=south east,xshift=-30pt,yshift=40pt]
              at (current page.south east) {
                \includegraphics[width=0.45\linewidth]{img/suricata.png}
              };
        \end{tikzpicture}%
        \begin{itemize}
            \item Software open source creato da\\OISF nel 2009, fork di Snort.
            \item Si può definire come IDPS,\\agisce sia da IDS che da IPS.
            \item Multipiattaforma.
        \end{itemize}
        
    \end{frame}
	\begin{frame}
    	\frametitle{Funzionalità extra di Suricata}
        \begin{itemize}
            \item Supporto al multithreading
            \item Uso di accelerazione grafica per l'analisi tramite tecnologia nVidia CUDA
            \item Trigger di script in linguaggio LUA
        \end{itemize}
      
    \end{frame}
 
    \begin{frame}
    	\frametitle{Installazione e configurazione}
		Installazione: seguire i passi del \href{https://suricata.readthedocs.io/en/suricata-6.0.4/quickstart.html\#installation}{sito della doc di Suricata}
        \\
        L'anno scorso avevamo una VM con Suricata con sopra Ubuntu.
        \\
        Lo script di installazione è presente \href{https://git.fuo.fi/fuomag9/docker-stuff-ipv6-ad/-/blob/main/scripts/install_suricata.sh}{qua}
    \end{frame}
    
    \begin{frame}
        \frametitle{Suricata.yaml}
        Nel file \texttt{suricata.yaml} vengono definiti vari parametri di funzionamento per il software e i percorsi di sistema per caricare le regole o salvare i file di log, oltre che per specificare determinati comportamenti.
        
        Solitamente si trova in \texttt{/etc/suricata/suricata.yaml}

    \end{frame}

    \begin{frame}[fragile]
        \frametitle{Campi importanti: pid-file}
        \texttt{pid-file}, se decommentata, definisce un percorso di default per il file
        contenente il pid di Suricata, utilizzato in assenza del comando \texttt{-{}-}\texttt{pidfile <file>}
        \vskip 0.3cm
        \begin{lstlisting}
# Default pid file.
# Will use this file if no --pidfile in command options.
# pid-file: /var/run/suricata.pid
        \end{lstlisting}
    \end{frame}
    
    \begin{frame}[fragile]
        \frametitle{Campi importanti: default log dir}
        \texttt{default-log-dir} specifica in quale directory Suricata andrà a scrivere i propri file di log.
        \vskip 0.3cm
        \begin{lstlisting}
default-log-dir: /var/log/suricata/
        \end{lstlisting}
    \end{frame}
    
    \begin{frame}[fragile]
        \frametitle{Campi importanti: default-rule-path}
        \texttt{default-rule-path} indica il percorso predefinito dove Suricata si aspetta di trovare i file \texttt{.rules}
        \vskip 0.3cm
        \begin{lstlisting}
# Set the default rule path here to search for the files.
# if not set, it will look at the current working dir
default-rule-path: /etc/suricata/rules
    \end{lstlisting}
    \end{frame}
    
    \begin{frame}[fragile]
        \frametitle{Campi importanti: rule-files}
        \texttt{rule-files} precede la lista di tutti i file di regole che Suricata caricherà all'avvio. I file qui indicati devono essere presenti nella directory specificata da \texttt{default-rule-path}. Qualora si volessero caricare regole esterne alla directory di default è possibile utilizzare il parametro \texttt{-s} quando si digita il comando Suricata.
        \vskip 0.3cm
        \begin{lstlisting}
[...]
- emerging-trojan.rules
- emerging-user_agents.rules
# - emerging-virus.rules
- emerging-voip.rules
- emerging-web_client.rules
- emerging-web_server.rules
[...]
        \end{lstlisting}
    \end{frame}
    \begin{frame}[fragile]
        \frametitle{Campi importanti: copy-mode}
    Consente di utilizzare Suricata come IPS.
    \vskip 0.3cm
    \begin{lstlisting}
netmap:
  - interface: wg-vulnbox
    copy-mode: ips
    \end{lstlisting}
    \end{frame}
    
    \begin{frame}
        \frametitle{Rules}
        Le rules sono delle istruzioni fornite all'IDPS che definiscono come il sistema debba comportarsi al presentarsi di determinati pacchetti o situazioni. Più regole inerenti un determinato argomento di sicurezza
        vengono raccolte all'interno di file con estensione \texttt{.rules}.
    \end{frame}
    
    \begin{frame}[fragile]
        \frametitle{Struttura delle regole}
        Le regole Suricata sono divise in tre parti:
        \begin{itemize}
            \item \textbf{Azione}: è l'azione che Suricata deve intraprendere al verificarsi delle condizioni definite a seguire nella regola.
            \item \textbf{Intestazione}: definisce il dominio di azione della regola, rendendone possibile l'innesco (trigger) solo quando un pacchetto cade all'interno di tale dominio.
            \item \textbf{Opzioni}: parole chiave che permettono di specificare ulteriormente il comportamento di una regola (ad esempio \texttt{msg}, \texttt{content}). 
        \end{itemize}
        \begin{lstlisting}
# Block OR 1=1
alert http any any -> any 8080 (msg:"OR 1=1 pcre"; content: "OR 1=1"; http_uri; nocase; sid: 100014; rev: 1;)
        \end{lstlisting}
    \end{frame}
    
    \begin{frame}
        \frametitle{Azioni}
        In caso di match con le regole
        \begin{itemize}
            \item \texttt{pass}: Suricata smette di esaminare il pacchetto e lo lascia passare.
            \item \texttt{alert}: il pacchetto viene lasciato passare, ma genererà un alert sotto forma di log. È l'azione più comune intrapresa nell'intrusion detection.
            \item \texttt{drop}: se Suricata agisce come IPS (modalità in-line)
            effettua il drop del pacchetto evitando che proceda oltre.
            Suricata genererà poi un alert per il pacchetto.
            \item \texttt{reject}: verrà inviato sia al destinatario che al mittente un pacchetto di reject (\texttt{reset-packet} per TCP, \texttt{ICMP-error} per altri protocolli). Se Suricata è disposto in modalità in-line, procederà al drop del pacchetto. Verrà generato un alert per il pacchetto.
        \end{itemize}
    \end{frame}
    
    \begin{frame}
    \frametitle{Intestazione}
    \begin{itemize}
        \item \textbf{Protocollo}: protocollo affinché la regola venga applicata (tcp, icmp, ip, http).
        \item \textbf{Sorgente e Destinazione}: Sono due coppie composte da Indirizzo IP di un host o di una rete e da una porta. Si possono usare:
        \begin{itemize}
            \item notazione CIDR
            \item parole chiave come \texttt{any}
            \item \texttt{\$HOME\_NET} e \texttt{\$EXTERNAL\_NET} definite in
            \texttt{suricata.yaml}.
        \end{itemize}
        \item \textbf{Direzione}: 
        \begin{itemize}
            \item \texttt{->}: \texttt{<IP, porta>} a sinistra funge da sorgente, mentre a destra si ha il destinatario
            \item \texttt{<-}: il contrario di \texttt{->}
            \item \texttt{<>}: entrambe le direzioni
        \end{itemize}
    \end{itemize}
    \end{frame}
    
    \begin{frame}[fragile]
        \frametitle{Regole di esempio}
        Con questi comandi Suricata carica delle regole pronte all'uso da enti che si occupano di sicurezza.
        \vskip 0.3cm
        \begin{lstlisting}
sudo suricata-update
sudo suricata-update list-sources
sudo suricata-update enable-source tgreen/hunting
        \end{lstlisting}
        \vskip 0.3cm
        \begin{lstlisting}
Name: et/open
  Vendor: Proofpoint
  Summary: Emerging Threats Open Ruleset
  License: MIT
Name: oisf/trafficid
  Vendor: OISF
  Summary: Suricata Traffic ID ruleset
  License: MIT
        \end{lstlisting}
    \end{frame}
    
    \begin{frame}
      \frametitle{Log Suricata}%
        Di default, Suricata, salva i propri output e alert su file di log. I vari log sono presenti nella cartella specificata nella variabile \texttt{default-log-dir} nel file \texttt{suricata.yaml}. 
        \vskip 0.3cm
        Su Windows genericamente viene usata la cartella \texttt{log} all'interno
        della cartella di installazione di Suricata. Su Linux viene spesso utilizzata la directory \texttt{/var/log/suricata}.
        \vskip 0.3cm
        Sotto la voce \texttt{outputs} del file di configurazione sono elencati i vari tipi di log che Suricata può salvare.
    \end{frame}
    
    \begin{frame}[fragile]
      \frametitle{Tipi di log: fast-log}%
        Sono log mono-riga, prodotto degli alert generati dalle regole di Suricata. È possibile scegliere se abilitare o meno questo tipo di log e se scrivere o meno in append al file. Questo genere di log si presentano con una specifica formattazione e contengono al loro interno quello che nelle regole è specificato dalla parola chiave \texttt{msg}.
        \vskip 0.3cm
        \begin{lstlisting}
12/03/2012-04:05:45.999712  [**] [1:1:1] Suricata! [**]
[Classification: Not Suspicious Traffic] [Priority: 3] {TCP}
192.168.1.2:58236 -> 91.198.174.224:80
        \end{lstlisting}
    \end{frame}
    
    \begin{frame}[fragile]
      \frametitle{Tipi di log: http-log}%
        Si tratta del file di log che registra le informazioni riguardo al traffico HTTP di passaggio attraverso il sensore. È possibile specificare il formato del log, se scrivere in append, e se abilitarlo o meno.
        \vskip 0.3cm
        \begin{lstlisting}
1/07/2012-01:55:30.337469 www.google.com [**]
/logos/Logo_25wht.gif [**] Mozilla/5.0 (X11; Ubuntu;
Linux x86_64; rv:16.0) Gecko/20100101 Firefox/16.0
[**] 90.147.13.136:46933 -> 173.194.35.177:80
64
        \end{lstlisting}
    \end{frame}
	\startlayoutpage
	\begin{frame}
		\begin{center}
			\Huge Domande?
		\end{center}
	\end{frame}
	\stoplayoutpage
\end{document}
